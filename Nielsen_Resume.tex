%%%%%%%%%%%%%%%%%%%%%%%%%%%%%%%%%%%%%%%%%%%%%%%%%%%%%%%%%%%%%%%%%%%%%%%%
%%%%%%%%%%%%%%%%%%%%%% Simple LaTeX CV Template %%%%%%%%%%%%%%%%%%%%%%%%
%%%%%%%%%%%%%%%%%%%%%%%%%%%%%%%%%%%%%%%%%%%%%%%%%%%%%%%%%%%%%%%%%%%%%%%%

%%%%%%%%%%%%%%%%%%%%%%%%%%%%%%%%%%%%%%%%%%%%%%%%%%%%%%%%%%%%%%%%%%%%%%%%
%% NOTE: If you find that it says                                     %%
%%                                                                    %%
%%                           1 of ??                                  %%
%%                                                                    %%
%% at the bottom of your first page, this means that the AUX file     %%
%% was not available when you ran LaTeX on this source. Simply RERUN  %%
%% LaTeX to get the ``??'' replaced with the number of the last page  %%
%% of the document. The AUX file will be generated on the first run   %%
%% of LaTeX and used on the second run to fill in all of the          %%
%% references.                                                        %%
%%%%%%%%%%%%%%%%%%%%%%%%%%%%%%%%%%%%%%%%%%%%%%%%%%%%%%%%%%%%%%%%%%%%%%%%

%%%%%%%%%%%%%%%%%%%%%%%%%%%% Document Setup %%%%%%%%%%%%%%%%%%%%%%%%%%%%

% Don't like 10pt? Try 11pt or 12pt
\documentclass[10pt]{article}

% This is a helpful package that puts math inside length specifications
\usepackage{calc}


% Simpler bibsection for CV sections
% (thanks to natbib for inspiration)
\makeatletter
\newlength{\bibhang}
\setlength{\bibhang}{1em}
\newlength{\bibsep}
 {\@listi \global\bibsep\itemsep \global\advance\bibsep by\parsep}
\newenvironment{bibsection}%
        {\vspace{-\baselineskip}\begin{list}{}{%
       \setlength{\leftmargin}{\bibhang}%
       \setlength{\itemindent}{-\leftmargin}%
       \setlength{\itemsep}{\bibsep}%
       \setlength{\parsep}{\z@}%
        \setlength{\partopsep}{0pt}%
        \setlength{\topsep}{0pt}}}
        {\end{list}\vspace{-.6\baselineskip}}
\makeatother

% Layout: Puts the section titles on left side of page
\reversemarginpar

%
%         PAPER SIZE, PAGE NUMBER, AND DOCUMENT LAYOUT NOTES:
%
% The next \usepackage line changes the layout for CV style section
% headings as marginal notes. It also sets up the paper size as either
% letter or A4. By default, letter was used. If A4 paper is desired,
% comment out the letterpaper lines and uncomment the a4paper lines.
%
% As you can see, the margin widths and section title widths can be
% easily adjusted.
%
% ALSO: Notice that the includefoot option can be commented OUT in order
% to put the PAGE NUMBER *IN* the bottom margin. This will make the
% effective text area larger.
%
% IF YOU WISH TO REMOVE THE ``of LASTPAGE'' next to each page number,
% see the note about the +LP and -LP lines below. Comment out the +LP
% and uncomment the -LP.
%
% IF YOU WISH TO REMOVE PAGE NUMBERS, be sure that the includefoot line
% is uncommented and ALSO uncomment the \pagestyle{empty} a few lines
% below.
%

%% Use these lines for letter-sized paper
\usepackage[paper=letterpaper,
            includefoot, % Uncomment to put page number above margin
            marginparwidth=1.2in,     % Length of section titles
            marginparsep=.05in,       % Space between titles and text
            margin=1in,               % 1 inch margins
            includemp]{geometry}

%% Use these lines for A4-sized paper
%\usepackage[paper=a4paper,
%            %includefoot, % Uncomment to put page number above margin
%            marginparwidth=30.5mm,    % Length of section titles
%            marginparsep=1.5mm,       % Space between titles and text
%            margin=25mm,              % 25mm margins
%            includemp]{geometry}

%% More layout: Get rid of indenting throughout entire document
\setlength{\parindent}{0in}

%% This gives us fun enumeration environments. compactitem will be nice.
\usepackage{paralist}

%% Reference the last page in the page number
%
% NOTE: comment the +LP line and uncomment the -LP line to have page
%       numbers without the ``of ##'' last page reference)
%
% NOTE: uncomment the \pagestyle{empty} line to get rid of all page
%       numbers (make sure includefoot is commented out above)
%
\usepackage{fancyhdr,lastpage}
%\pagestyle{fancy}
\pagestyle{empty}      % Uncomment this to get rid of page numbers
\fancyhf{}\renewcommand{\headrulewidth}{0pt}
\fancyfootoffset{\marginparsep+\marginparwidth}
\newlength{\footpageshift}
\setlength{\footpageshift}
          {0.5\textwidth+0.5\marginparsep+0.5\marginparwidth-2in}
\lfoot{\hspace{\footpageshift}%
       \parbox{4in}{\, \hfill %
                    \arabic{page} of \protect\pageref*{LastPage} % +LP
%                    \arabic{page}                               % -LP
                    \hfill \,}}

% Finally, give us PDF bookmarks
\usepackage{color,hyperref}
\definecolor{darkblue}{rgb}{0.0,0.0,0.3}
\hypersetup{colorlinks,breaklinks,
            linkcolor=darkblue,urlcolor=darkblue,
            anchorcolor=darkblue,citecolor=darkblue}

%%%%%%%%%%%%%%%%%%%%%%%% End Document Setup %%%%%%%%%%%%%%%%%%%%%%%%%%%%


%%%%%%%%%%%%%%%%%%%%%%%%%%% Helper Commands %%%%%%%%%%%%%%%%%%%%%%%%%%%%

% The title (name) with a horizontal rule under it
%
% Usage: \makeheading{name}
%
% Place at top of document. It should be the first thing.
\newcommand{\makeheading}[3]%
        {\hspace*{-\marginparsep minus \marginparwidth}%
         \begin{minipage}[t]{\textwidth+\marginparwidth+\marginparsep}%
                {\large \bfseries \hspace*{185 pt} #1}\\{\bfseries #2 \hspace*{260 pt} #3}\\[-0.15\baselineskip]%
                 \rule{\columnwidth}{1pt}%
         \end{minipage}}

% The section headings
%
% Usage: \section{section name}
%
% Follow this section IMMEDIATELY with the first line of the section
% text. Do not put whitespace in between. That is, do this:
%
%       \section{My Information}
%       Here is my information.
%
% and NOT this:
%
%       \section{My Information}
%
%       Here is my information.
%
% Otherwise the top of the section header will not line up with the top
% of the section. Of course, using a single comment character (%) on
% empty lines allows for the function of the first example with the
% readability of the second example.
\renewcommand{\section}[2]%
        {\pagebreak[2]\vspace{1.3\baselineskip}%
         \phantomsection\addcontentsline{toc}{section}{#1}%
         \hspace{0in}%
         \marginpar{
         \raggedright \scshape #1}#2}

% An itemize-style list with lots of space between items
\newenvironment{outerlist}[1][\enskip$\circ$]%
        {\begin{itemize}[#1]}{\end{itemize}%
         \vspace{-.6\baselineskip}}

% An environment IDENTICAL to outerlist that has better pre-list spacing
% when used as the first thing in a \section
\newenvironment{lonelist}[1][\enskip\textbullet]%
        {\vspace{-\baselineskip}\begin{list}{#1}{%
        \setlength{\partopsep}{0pt}%
        \setlength{\topsep}{0pt}}}
        {\end{list}\vspace{-.6\baselineskip}}

% An itemize-style list with little space between items
\newenvironment{innerlist}[1][\enskip$\circ$]%
        {\begin{compactitem}[#1]}{\end{compactitem}}

% An environment IDENTICAL to innerlist that has better pre-list spacing
% when used as the first thing in a \section
\newenvironment{loneinnerlist}[1][\enskip\textbullet]%
        {\vspace{-\baselineskip}\begin{compactitem}[#1]}
        {\end{compactitem}\vspace{-.6\baselineskip}}

% To add some paragraph space between lines.
% This also tells LaTeX to preferably break a page on one of these gaps
% if there is a needed pagebreak nearby.
\newcommand{\blankline}{\quad\pagebreak[2]}

% Uses hyperref to link DOI
\newcommand\doilink[1]{\href{http://dx.doi.org/#1}{#1}}
\newcommand\doi[1]{doi:\doilink{#1}}

% For \url{SOME_URL}, links SOME_URL to the url SOME_URL
\providecommand*\url[1]{\href{#1}{#1}}
% Same as above, but pretty-prints SOME_URL in teletype fixed-width font
\renewcommand*\url[1]{\href{#1}{\texttt{#1}}}

% For \email{ADDRESS}, links ADDRESS to the url mailto:ADDRESS
\providecommand*\email[1]{\href{mailto:#1}{#1}}
% Same as above, but pretty-prints ADDRESS in teletype fixed-width font
%\renewcommand*\email[1]{\href{mailto:#1}{\texttt{#1}}}

%%%%%%%%%%%%%%%%%%%%%%%% End Helper Commands %%%%%%%%%%%%%%%%%%%%%%%%%%%

%%%%%%%%%%%%%%%%%%%%%%%%% Begin CV Document %%%%%%%%%%%%%%%%%%%%%%%%%%%%

\begin{document}
\makeheading{David L.~Nielsen}{208-310-0902}{\email{david.nielsen88@gmail.com}
}
%\section{Contact Information}
%%
%% NOTE: Mind where the & separators and \\ breaks are in the following
%%       table.
%%
%% ALSO: \rcollength is the width of the right column of the table
%%       (adjust it to your liking; default is 1.85in).
%%
%\newlength{\rcollength}\setlength{\rcollength}{2.5in}%
%%
\smallbreak
\section{Education}
%
\href{http://www.compling.uw.edu/}{\textbf{University of Washington}},
Seattle, Washington
\hfill{Currently enrolled}
\begin{innerlist}
\item  MS candidate in Computational Linguistics.
\item Expanding my machine learning knowledge to include Natural Language Processing techniques such as text classification, POS tagging, feature creation using deep linguistic structures, and word embeddings.

\end{innerlist}
\smallbreak

\href{http://www.reed.edu/}{\textbf{Reed College}},
Portland, Oregon
\hfill{August 2008-May 2011}
\begin{innerlist}
\item B.A., Mathematics, GPA 3.48
% \item Received two institutional scholarships and an academic commendation for excellence. 
\item Senior thesis: \textit{The Problem of Zarankiewicz} is a mathematical puzzle concerning the  number of shared connections between two groups. Discovered and proved a new lower bound for certain cases of the problem.
%\item Carmine Award 2011 for leadership in the Reed College Rugby program, two institutional scholarships and academic commendation for excellence.       
\end{innerlist}

%\href{http://www.byu.edu/}{\textbf{Brigham Young University}},
%Provo, Utah
%\hfill{ August 2007-May 2008}
%\begin{innerlist}
%\item Awarded full merit scholarship.
%\end{innerlist}



\section{Work Experience}
\href{http://www.tura.io}{\textbf{Tura.io}, Portland, Oregon}
\begin{outerlist}
\item[] \textit{Software Architect} \hfill{Sept 2017 to present}
\begin{innerlist}
\item Lead our work designing a library for complex pattern recognition in real-time IoT data streams.
\item Prototyped a platform to store, manage, recall and re-evaluate the output from this pattern recognition in a scalable cloud environment. 
\item  Followed agile software developing practices and developed primarily in {\sc Python} and {\sc TensorFlow} on {\sc Ubuntu} and with {\sc Docker} to containerize our platform.
\end{innerlist}
\end{outerlist}
\medbreak

\href{http://ti.arc.nasa.gov/}{\textbf{Intelligent Systems Division, NASA Ames Research Center}, California}
\begin{outerlist}
\item[] \textit{Research Engineer}, \href{http://www.moriassociates.com/}{\textbf{MORi Associates, Inc.}} % 
\hfill{Feb 2013 to Oct 2016}

  \begin{innerlist}
    \item Collaborated with the Data Sciences group to discover, explain, and predict safety and operational incidents in aviation using data mining and machine learning techniques.
    \item Created novel anomaly detection algorithms to discover and investigate landings at four of the largest US airports. These algorithms improved upon the state of the art machine learning techniques and the results were published in {\sc IEEE Digital Avionics Systems Conferences} and {\sc 2016 World Congress on Computational Intelligence}.
    \item Developed using the {\sc Python} scientific stack ({\sc NumPy, SciPy, Scikit-Learn}) in a  {\sc Linux} environment with {\sc git} for version control. I also built {\sc Shell Scripts} for data management on our local network.

\end{innerlist}
\end{outerlist}
\medbreak
\href{http://www.ogi.edu/bme}{\textbf{Department of Biomedical Engineering, \\Oregon Health and Science University}},
Portland, Oregon
\begin{outerlist}
\item[] \textit{Research Assistant}%
    \hfill {July 2011 to June 2012}
     \begin{innerlist}
     \item Collaborated with Dr. Todd Leen on perturbation methods for statistical analysis of neural modeling..
     \item By applying these techniques we were able to approximate  both online learning algorithms and electric pulses generated by fish which much higher confidence that before, leading to the publication of peer-reviewed journal articles.
     \end{innerlist}
   \end{outerlist}
  

%\smallbreak
%\section{Technical\ \  Skills}
%\begin{innerlist} 
%\item{} I have experience in {\sc Machine Learning} with focus on unsupervised methods for anomaly detection in heterogenous data using {\sc SVM} and {\sc ELM}.
%\item{} Within the {\sc Natural Language Processing} domain, I have worked on text classification, POS tagging, supervised feature creation using deep linguistic structures, and unsupervised embedding generation.
%\item{} I primarily develop using the {\sc Python} scientific stack ({\sc NumPy, SciPy, Scikit-Learn, NLTK, TensorFlow}) in a  {\sc Linux} environment ({\sc Ubuntu, RedHat, Raspbian, Shell Scripts}) with {\sc git} for version control and {\sc Docker} for containerization.   
%\item{Languages: }  {\sc Python, NumPy, Scikit-Learn, Matlab, Shell Scripts, SQL}
%\item {Software: } {\sc Spyder, git, SVN}
%\item {Operating Systems: } {\sc RedHat Linux, Raspbian, OSX, Windows}
%\item{Databases: } {\sc PostgreSQL}
%\end{innerlist}
%\smallbreak
%\section{Volunteer Experience}
%\href{http://bikex.org/}{\textbf{Silicon Valley Bicycle Exchange}}
%, Mountain View, California
%\begin{outerlist}
%\item[] \textit{Mechanic/Mentor}%
%  \hfill{March 2013 to present}
%  \begin{innerlist}
%  \item Repaired bikes for donation to those in need including veterans groups, day worker centers and local schools.
%  \item  Served as mentor to new volunteers while working on bicycles together.
%  \end{innerlist}
%  
%\end{outerlist}
%\bigbreak

%\section{Honors}
%Eagle Scout, Member of National Honor Society, Full scholarship at BYU, James B. Small Scholarships 2009-2010, Henry W. Wessinger Memorial Scholarship 2010-2011, Reed College academic commendation for excellence 2010-2011, Carmine Award 2011 for teaching and contributing to the Reed College Rugby program.



\begin{outerlist}
\item[]References available upon request. 
\end{outerlist}


\end{document}

%%%%%%%%%%%%%%%%%%%%%%%%%% End CV Document %%%%%%%%%%%%%%%%%%%%%%%%%%%%%
